\section{Appendix: RCA and RpcA}%
\label{sec:rca}
Most of the early papers in economic complexity (e.g \citealp{tacchella_new_2012,hidalgo_building_2009}) used the revealed comparative advantage \citep{balassa_trade_1965} to normalize trade data. Revealed comparative advantage is defined as

\[
	RCA_{cp} = \frac{X_{cp}}{\sum_p X_{cp}} \bigg / \frac{\sum_c X_{cp}}{\sum_{cp} X_{cp}}
\]

where $X_{cp}$ is the export value of country $c$'s export of product $p$.

There are three main drawbacks of using RCA:

\begin{enumerate}
\item While in practice used to confer an idea of what countries are good at, RCA in fact measures the relative intensity of countries' exports. For instance, consider if a country is very poor at all types of production. This means its total exports will be low. If the country is just a little bit less hapless at one product, this product will have a very high RCA, since other countries will have less of a gulf between their export basket and this one given product. This, despite it being less proficient in producing the product than other countries.
\item Secondly, if a country exports a lot of something very valuable - like oil - and it dwarfs the contribution of other products to the value of the country's total exports, this product will dominate the first term in the RCA formula ($X_{cp} / \sum_p X_{cp}$). Even if the country is very good at producing other products, their RCA will be artificially lowered by the valuable commodity.
\item Finally, the RCA of a product is connected to the prices of other products. This means that should a country export three products, and the price of one of them falls sharply, the RCA in the two other products rises for the country. Should a second country be equally proficient in exporting the two products but their third product's price stay put, they will now seem worse a exporting the two products.
\end{enumerate}

A different metric, used in a few of the newer papers \citep{hausmann_implied_2019}, instead measures the per-capita export in a product normalized by global (total) per capita export in the product. The "revealed per capita advantage" (RpcA) for country $c$ in $p$ is then simply defined as:

\[
	RpcA_{cp} = \frac{X_{cp}}{POP_{c}} \bigg / \frac{\sum_c X_{cp}}{\sum_c POP_c}
\]

where $POP_c$ is the population of country $c$.

By not including the total export basket of country $c$ in the normalization we avoid the three issues outlined above, and catch a more "absolute" skill based measure of the economy's capabilities. 

\subsection{Distribution of comparative advantage}
\label{subsec:dist}

Changing from RCA to RpcA has some distinct impacts on the distribution of comparative advantage, both in terms of the total distribution and in which countries are more ``significant exporters''. Figure \ref{fig:rca_rpca_hist} shows the distribution of countries by the number of products they are significant exporters of by RCA (\ref{fig:rca_hist}) and RpcA (\ref{fig:rpca_hist}). Using RpcA results in a much more right-skewed distribution, with some countries exporting almost all available products.

% FIGURE: rca vs rpca histograms
\begin{figure}
     \centering
     \begin{subfigure}[b]{0.45\textwidth}
         \centering
         \includegraphics[width=\textwidth]{figures/appendix/appendix_histogram_rca}
         \caption{RCA}
         \label{fig:rca_hist}
     \end{subfigure}
     \hfill
     \begin{subfigure}[b]{0.45\textwidth}
         \centering
         \includegraphics[width=\textwidth]{figures/appendix/appendix_histogram_rpca}
         \caption{RpcA}
         \label{fig:rpca_hist}
     \end{subfigure}
        \caption{Distribution of countries by number of products with
		  comparative advantage (across sample years).}
        \label{fig:rca_rpca_hist}
\end{figure}

One of the most interesting changes when shifting to RpcA is in the relationship to total population. When using RpcA the correlation between population and the number of products exported by a country disappears completely. Figure \ref{fig:rca_rpca_by_pop} shows the relationship across the sample years. It can be a bit hard to see what is going on, so the histogram in figure \ref{fig:p_val_hist_pop} shows the distribution of p-values for the population term in a simple linear regression on country fitness by the natural log of population. Population is not significant for any year using RpcA but for every year using RCA. Next to it, figure \ref{fig:stand_diff_fit_pop} shows the relationship between the difference in fitness values based on RCA and RpcA and population. Differences are taken by first standardizing fitness values to \(F^{z}_{c}\) by

\[
 F^{z}_{c} = \frac{F_{c} - <F_{c}>}{sd(F_{c})}
\]

where \(F_{c}\) is the fitness of country \(c\) in the given year, and then subtracting the \(F^{z}_{c}\) based on RpcA from the one based on RCA. That is, a negative difference means that fitness for a given county is higher using fitness calculated from RpcA. The result, which are especially driven by China and India, shows that countries that have a larger population tend to have their fitness reduced more by changing the trade-normalization metric. The result is not driven by penalizing large populations, but by removing the positive relationship between comparative advantage and population.

% FIGURE: rca, rpca by pop all years
\begin{figure}
     \centering
     \begin{subfigure}[b]{0.45\textwidth}
         \centering
         \includegraphics[width=\textwidth]{figures/appendix/appendix_rca_by_pop}
         \caption{RCA by population, ln}
         \label{fig:rca_by_pop}
     \end{subfigure}
     \hfill
     \begin{subfigure}[b]{0.45\textwidth}
         \centering
         \includegraphics[width=\textwidth]{figures/appendix/appendix_rpca_by_pop}
         \caption{RpcA by population, ln}
         \label{fig:rpca_by_pop}
     \end{subfigure}
	 \caption{Relationship between ln(population) and the number of products
	   with comparative advantage in a country (across sample years).}
        \label{fig:rca_rpca_by_pop}
\end{figure}

% FIGURE: P-val hist and stand diff 2010 by pop
\begin{figure}
     \centering
     \begin{subfigure}[b]{0.45\textwidth}
         \centering
         \includegraphics[width=\textwidth]{figures/appendix/appendix_p_val_hist_rca_rpca_by_pop}
         \caption{p-values of ln(pop) term on fitness}
         \label{fig:p_val_hist_pop}
     \end{subfigure}
     \hfill
     \begin{subfigure}[b]{0.45\textwidth}
       \centering
         \includegraphics[width=\textwidth]{figures/appendix/appendix_fitness_difference_by_pop_2010}
         \caption{difference in fitness by ln(pop) (2010), spearman correlation and SLR}
         \label{fig:stand_diff_fit_pop}
     \end{subfigure}
	 \caption{Relationship between ln(population) and the number of products with comparative advantage in a country (across sample years).}
        \label{fig:population_difference}
\end{figure}

\subsection{Significance for complexity metrics}
\label{subsec:significance_for_complexity}

The change in comparative advantage clearly changes the $M$ matrix used in the fitness algorithm. This means that the fitness of countries change as well. Figure \ref{fig:bar_plot_diff} show the change in standardized fitness when using RCA and RpcA in 2010. The figure shows a similar picture to \ref{fig:stand_diff_fit_pop}: more populous countries tend to be less fit using RpcA. Figure \ref{fig:rca_rpca_fit_hist} shows the distribution of fitness values across the sample years. Unsurprisingly, they are similar to the distribution of comparative advantage in figure \ref{fig:rca_rpca_hist}: RpcA-based fitness are more right-skewed.

TODO: Volatility. Given that population of countries are less volatile over time than commodity prices, we would expect that product complexity values are more stable from year to year using RCA than RpcA.
% FIGURE: difference 2010 bar
\begin{figure}[ht]
  \centering
  \includegraphics[width=\textwidth]{figures/appendix/appendix_fitness_difference_bar}
  \caption{Difference in country fitness when changing normalization metric from RCA to RpcA (2010-data). Negative values have a higher fitness value when using RpcA than RCA as normalization metric.}
  \label{fig:bar_plot_diff}
\end{figure}

% FIGURE: rca vs rpca histograms
\begin{figure}
     \centering
     \begin{subfigure}[b]{0.45\textwidth}
         \centering
         \includegraphics[width=\textwidth]{figures/appendix/appendix_fitness_rca_histogram}
         \caption{RCA-based fitness}
         \label{fig:rca_fit_hist}
     \end{subfigure}
     \hfill
     \begin{subfigure}[b]{0.45\textwidth}
         \centering
         \includegraphics[width=\textwidth]{figures/appendix/appendix_fitness_rpca_histogram}
         \caption{RpcA-based fitness}
         \label{fig:rpca_fit_hist}
     \end{subfigure}
        \caption{Distribution of countries by fitness values (across sample years).}
        \label{fig:rca_rpca_fit_hist}
\end{figure}

% Importance for oveall distribution of fitness
% Importance for population
% Importance for resource rents
% biggest losers, winners

%%%%%%%%%%%%%%%%%%%%%%%%%%%%%%%%%%%%%%%%%%%%%%%%%%%%%%%%%%%%%%%%%%%%%%%%%%%%%%%%
%%%%%%%%%%%%%%%%%%%%%%%%%%%%%%%%%%%%%%%%%%%%%%%%%%%%%%%%%%%%%%%%%%%%%%%%%%%%%%%%

\newpage

\section{Appendix: Which algorithm? Fitness-Complexity and Hausmann-Hidalgo}
\label{sec:appendix-algorithm}

% FIGURE: convergence of fitness, complexity
\begin{figure}
     \centering
     \begin{subfigure}[b]{0.45\textwidth}
         \centering
         \includegraphics[width=\textwidth]{figures/appendix/appendix_fitness_convergence_plot}
	 \caption{Convergence of fitness values}
         \label{fig:fit_conv}
     \end{subfigure}
     \hfill
     \begin{subfigure}[b]{0.45\textwidth}
         \centering
         \includegraphics[width=\textwidth]{figures/appendix/appendix_complexity_convergence_plot}
         \caption{Convergence of complexity values}
         \label{fig:comp_conv}
     \end{subfigure}
     \caption{Each line represents a country or a product in 2010. For most of the countries or products the FC algorithm reaches its fixed point after relatively few iterations. Note that lines are colored by the actual value, not the natural log.}
        \label{fig:fc_conv}
\end{figure}


TODO
\end{appendices}

\newpage
% Table created by stargazer v.5.2.2 by Marek Hlavac, Harvard University. E-mail: hlavac at fas.harvard.edu
% Date and time: Wed, May 20, 2020 - 19:41:52
 

%%%%%%%%%%%%%%%%%%%%%%

\subsection{Cleaning international trade data}%l
\label{sub:cleaning_international_trade_data}

The data in international trade comes the BACI database maintained by CEPII (build on raw data from UN COMTRADE). CEPII cleans the data using the their own methodology. They exploit that a trade flow is usally reported by both the importer and the exporter. This makes it possible to assign a reliability ranking to each reporter, which can be used to give weights to different soruces of information, giving better export observations \citep{gualier_baci_2010}. 

To minimize the year-to-year noise, I remove city-sized economies and the smallest exporters. To be included in the final sample, an economy must have a population of at least 1 million in 2005 and export for at least 1 billion current USD in 2005. As for the choice of reference year, the prerable year would have been the sample mid-year, around 2008, but the financial crash leaves exports unrepresentative. The sample is not very sensitive to changing the reference year. 

In addition, I remove a few countries that supply highly unreliable trade information (Iraq, Afghanistan, and Chad) and a few products that are not being exported at all during the sample (9704 \footnote{"Stamps, postage or revenue; stamp-postmarks, first-day covers, postal stationery (stamped paper) and like, used, or if unused not of current or new issue in the country to which they are destined"}, 2527 \footnote{"Natural cryolite; natural chiolite"}, 1403 \footnote{"Vegetable materials of a kind used primarily in brooms or brushes; (eg broomcorn, piassava, couch-grass and istle), whether or not in hanks or bundles"}). All products that are exported by a country for less than 1000 dollars in a given year are set to 0. 

The original export data covers 221 countries and "country-like" regions. After cleaning, 120 countries remains. These remaining countries covers 92 percent of the total export in the raw data, 94 percent of population in the original data, and 98 percent of the GDP in the original data. 

%%%%%%%%%%%%%%%%%%%%%%%



TODO: BRUG PCI istedet


    I use the Fitness-Complexity (FC) algorithm \citep{tacchella_new_2012} to find the complexity of products. For a discussion of the benefits of the FC algorithm over the  Hausmann-Hidalgo (HH) algorithm \citep{hidalgo_building_2009}, see appendix \ref{sec:appendix-algorithm}. As outlined earlier, countries, capabilities and products are connected in a tripartite network. The capabilities in this network, however, is hidden.

I first find the revealed per capita advantage (RpcA) of each country in each of the approx. 1200 products in the HS92 series. For a discussion of why RpcA is an improvement over the more often used revealed comparative advantage (RCA, \citealp{balassa_trade_1965}) see appendix \ref{sec:rca}. RpcA normalizes a country's per capita export of a product by the global per capita export of the product. Hence, RpcA of country \(c\) in product \(p\):

\[
	RpcA_{cp} = \frac{X_{cp}}{POP_{c}} \bigg / \frac{\sum_c X_{cp}}{\sum_c POP_c}
\]

where \(X_{cp}\) is the export value of country \(c\) in product \(p\) and \(POP_{c}\) is the population of country \(p\). I then define a binary RpcA matrix \(M_{cp}\) with countries in rows as products in columns as:

\[
M_{cp} = \begin{cases}
 1 & \text{if } RpcA_{cp} \geq 1 \\
 0 & \text{if } RpcA_{cp} < 1
\end{cases}
\]

This \(M_{cp}\) matrix can be interpreted as a bipartite network where countries are connected to the products they export competatively. This is the observed model in figure \ref{fig:complexity-model}. 

TODO about the FC algorithm 

The fitness of a country, \(F_{c}\) is the sum of all the products it exports, weighted by their complexity, \(Q_{p}\). For each iteration there are two steps. First I find the temporary variables \(\hat{F}^{(n)}_{c}\) and \(\hat{Q}^{(n)}_{p}\), next they are normalized by the average value of the iteration. This normalization procedure means that after enough iterations, \(F_{c}\) and \(Q_{p}\) converges to a fixed point (see figure \ref{fig:fc_conv}), meaning that the initial conditions are not important. I set them to 1 for all \(\hat{F}^{(0)}_{c}\) and \(\hat{Q}^{(0)}_{p}\).

 \[
	 \begin{split}
		 \hat{F}^{(n)}_{c} &= \sum_p M_{cp} Q^{(n-1)}_{p} \Rightarrow F^{(n)}_{c} = \frac{\hat{F}^{(n)}_{c}}{\bigg < \hat{F}^{(n)}_c \bigg > _c} \\
		 \hat{Q}^{(n)}_{p} &= \frac{1}{\sum_c M_{cp} \frac{1}{F^{(n-1)}_c}} \Rightarrow Q^{(n)}_{p} = \frac{\hat{Q}^{(n)}_{p}}{\bigg < \hat{Q}^{(n)}_p \bigg > _p }
	 \end{split}
\]

TODO: One of the issues...

TODO: On section on HS96 four digit products, give example of how specific products are.
TODO: Lav sanity check if antallet af forskellige producter hvert år i orignal listing og antallet af forskellige producter i hvert år efter concordance.
TODO: BRUG PCI istedet


    I use the Fitness-Complexity (FC) algorithm \citep{tacchella_new_2012} to find the complexity of products. For a discussion of the benefits of the FC algorithm over the  Hausmann-Hidalgo (HH) algorithm \citep{hidalgo_building_2009}, see appendix \ref{sec:appendix-algorithm}. As outlined earlier, countries, capabilities and products are connected in a tripartite network. The capabilities in this network, however, is hidden.

I first find the revealed per capita advantage (RpcA) of each country in each of the approx. 1200 products in the HS92 series. For a discussion of why RpcA is an improvement over the more often used revealed comparative advantage (RCA, \citealp{balassa_trade_1965}) see appendix \ref{sec:rca}. RpcA normalizes a country's per capita export of a product by the global per capita export of the product. Hence, RpcA of country \(c\) in product \(p\):

\[
	RpcA_{cp} = \frac{X_{cp}}{POP_{c}} \bigg / \frac{\sum_c X_{cp}}{\sum_c POP_c}
\]

where \(X_{cp}\) is the export value of country \(c\) in product \(p\) and \(POP_{c}\) is the population of country \(p\). I then define a binary RpcA matrix \(M_{cp}\) with countries in rows as products in columns as:

\[
M_{cp} = \begin{cases}
 1 & \text{if } RpcA_{cp} \geq 1 \\
 0 & \text{if } RpcA_{cp} < 1
\end{cases}
\]

This \(M_{cp}\) matrix can be interpreted as a bipartite network where countries are connected to the products they export competatively. This is the observed model in figure \ref{fig:complexity-model}. 

TODO about the FC algorithm 

The fitness of a country, \(F_{c}\) is the sum of all the products it exports, weighted by their complexity, \(Q_{p}\). For each iteration there are two steps. First I find the temporary variables \(\hat{F}^{(n)}_{c}\) and \(\hat{Q}^{(n)}_{p}\), next they are normalized by the average value of the iteration. This normalization procedure means that after enough iterations, \(F_{c}\) and \(Q_{p}\) converges to a fixed point (see figure \ref{fig:fc_conv}), meaning that the initial conditions are not important. I set them to 1 for all \(\hat{F}^{(0)}_{c}\) and \(\hat{Q}^{(0)}_{p}\).

 \[
	 \begin{split}
		 \hat{F}^{(n)}_{c} &= \sum_p M_{cp} Q^{(n-1)}_{p} \Rightarrow F^{(n)}_{c} = \frac{\hat{F}^{(n)}_{c}}{\bigg < \hat{F}^{(n)}_c \bigg > _c} \\
		 \hat{Q}^{(n)}_{p} &= \frac{1}{\sum_c M_{cp} \frac{1}{F^{(n-1)}_c}} \Rightarrow Q^{(n)}_{p} = \frac{\hat{Q}^{(n)}_{p}}{\bigg < \hat{Q}^{(n)}_p \bigg > _p }
	 \end{split}
\]

TODO: One of the issues...

TODO: On section on HS96 four digit products, give example of how specific products are.
TODO: Lav sanity check if antallet af forskellige producter hvert år i orignal listing og antallet af forskellige producter i hvert år efter concordance.
