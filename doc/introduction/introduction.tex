\section{Introduction and background}\label{sec:introduction_and_background}

The products produced in rich countries are very different from the products
produced in poor countries. Why? The difference between countries' economic
sophistication expressed through production baskets explains large variation in
GDP per capita (see figure \ref{fig:framework-eci-gdp}) and growth rates \citep{tacchella_dynamical_2018}. Despite this, little empirical evidence examines the micro-foundations behind these differences.

The study outlined in this paper examines the ability of (electricity) interruptions in the production environment to explain the complexity of products made at the plant level in India between 1995 and 2010. The mechanism is simple. If more complex products require more intermediate inputs or more steps in production, interruptions or failures in production processes punish them harder than simpler ones. If these disruption-costs are punitive enough, investors will put their money elsewhere and producers will choose less complex products.

\begin{figure}[htpb]
	\centering
	\includegraphics[width=0.8\linewidth]{figures/introduction_disruption_plot.pdf}
	\caption{Production disruptions tend to have a higher frequence in poorer, less complex economies. The color of points are country fitness values (gray points has missing values). Lines are SLR fits to ln(GDP/cap). Disruption data is from \cite{the_world_bank_enterprise_2019} and fitness is constructed as outline in section \ref{sec:product_complexity}. Countries are not surveyed the same years, so fitness and GDP/cap values have been matched to the newest survey year for each country before than 2016. Years range between 2008 and 2015. No country has more than one observation.}
	\label{fig:disruption-fig}
\end{figure}

The role electricity in production inefficiencies have recently drawn the attention of researchers. Two India-specific patterns are worth highlighting: the importance of the quality of electricity, rather than just electricity alone, and the evidence on output-costs of electricity shortages. \cite{samad_benefits_2016} and \cite{chakravorty_does_2014} both find improvements in incomes from access to electricity, but much larger income gains from access to quality electricity (from 9.6\% to 17 \% and from 9\% to 28.6\%)\footnote{The latter study only non-agricultural incomes.}. \cite{abeberese_electricity_2017} finds that increases in electricity prices reduces the electricity- and machine-intensity of activities, leading to smaller growth in production output. Finally, \cite{allcott_how_2016} finds a 5\% - 10\% reduction in plant-level profits from shortages in electricity, with a lower productivity-penalty due to generator substitution. The approach taken in this paper proposes a way to unify these observations. Unreliable electricity punishes more complex plants and improvements in quality have positive externalities on the local production, and a more complex manufacturing sector has positive effects on local incomes. Similarly, the relatively small output-penalty observed by \cite{allcott_how_2016} could mask production choices into simpler and lower-complexity products.

The approach taken in this paper is primarily related to three strands of literature. First, the literature of economic complexity and the sophistication of an economy's capability base  (\citealp{frenken_related_2007,hausmann_atlas_2013,tacchella_new_2012}). In addition, to develop the relation between production disruptions and product complexity, I draw on O-ring-type effects as modeled in \cite{kremer_o-ring_1993} and \cite{jones_intermediate_2011}, and the recent literature on volatility in production networks \citep{acemoglu_network_2012}. While the latter is mainly concerned with aggregate effect of sectoral shocks, the importance of input-output linkages are equally relevant at the plant level.

The rest of the paper goes as follows. I first develop the conceptual framework that underpins the work here: why should we care about economic complexity (\ref{sec:frame-product-complexity}) and how are electricity shortages related to whether factories produce more or less complex products (\ref{sec:production_networks})? From the conceptual framework I draw out two testable hypotheses on the proposed relationship. Building from these hypotheses, I turn to the research design, my empirical strategy, and the use of data. [The rest is TODO]
